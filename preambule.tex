\documentclass[russian,utf8,pointsection]{eskdtext}
\usepackage{cmap}			% Поиск по русским словам в конечном pdf документе
\usepackage{eskdchngsheet}
\usepackage[T2A]{fontenc}
\usepackage{pscyr}			% Подключение "красивых" шрифтов киррилицы
\usepackage{amstext}
\usepackage{amsmath}
\usepackage{listings}

\usepackage{listings}		
\lstset{ %
	language=tcl,           % Язык программирования 	
	frame=single,           % Добавить рамку
	breaklines=true,        % Автоматический перенос строк
	breakatwhitespace=true, % Переносить строки по словам
	title=\lstname 
}

\usepackage{pdflscape}

% работа с сылками
\usepackage{hyperref}
\usepackage[usenames,dvipsnames,svgnames,table,rgb]{xcolor}
\hypersetup{			
	unicode=true,           				% русские буквы в раздела PDF
	pdftitle={Заголовок},   				% Заголовок
	pdfauthor={Автор},      				% Автор
	pdfsubject={Тема},      				% Тема
	pdfcreator={Создатель}, 				% Создатель
	pdfproducer={Производитель}, 			% Производитель
	pdfkeywords={keyword1} {key2} {key3}, 	% Ключевые слова
	colorlinks=true,      					% false: ссылки в рамках; true: цветные ссылки
	linkcolor=NavyBlue,        				% внутренние ссылки
	citecolor=black,        				% на библиографию
	filecolor=black,        				% на файлы
	urlcolor=black          				% на URL
}

% Дает доступ к командам:
% \MakeTextUppercase{} - сделать все символы заглавными
\usepackage{textcase} 

%Изменение отображения Содержания
%\makeatletter
%\renewcommand{\l@section}{\@dottedtocline{1}{0em}{1.25em}}
%\renewcommand{\l@subsection}{\@dottedtocline{2}{1.25em}{1.75em}}
%\renewcommand{\l@subsubsection}{\@dottedtocline{3}{2.75em}{2.6em}}
%\makeatother

% Работа с таблицами
% p{} - top align, m{} - middle align, b{} - bottom align
\usepackage{ltablex} 										% longtable с функциональностю tabularx
\usepackage{multirow} 										% Слияние строк в таблице
\renewcommand{\tabularxcolumn}[1]{>{\arraybackslash}m{#1}}	% выравнивание в ячейке таблицы по середине по вертикали
\newcolumntype{M}[1]{>{\centering \arraybackslash}m{#1}} 	% колонка с заданной шириной и выравниванием по центру
\newcolumntype{Z}{>{\centering \arraybackslash} X} 			% колонка с выравниванием по центру

% Добавлено отображение Города на Титульном листе
\renewcommand{\ESKDtheTitleFieldX}{Екатеринбург \\ \ESKDtheYear}

% Уменьшен размер шрифта для заголовков секций
\ESKDsectStyle{section}{\large \bfseries \MakeTextUppercase}

% Выравнивание по центру.
% Предназначено для выравнивания надписей в шапке таблицы
\newcommand{\calign}[1]{\centering #1 \arraybackslash} 

%%% Работа с картинками
\usepackage{graphicx}  			% Для вставки рисунков
\graphicspath{{section/images/}}  % папки с картинками
\setlength\fboxsep{3pt} 		% Отступ рамки \fbox{} от рисунка
\setlength\fboxrule{1pt} 		% Толщина линий рамки \fbox{}
\usepackage{wrapfig} 			% Обтекание рисунков и таблиц текстом
\usepackage{float}

%%% Информация
\newcommand{\info}[1]{ %
\begin{flushleft}%
	\begin{tabular*}{\linewidth}{m{0.05\linewidth}m{0.9\linewidth}}%
		\includegraphics[width=\linewidth]{info.jpg} & \textcolor{ForestGreen}{\textit{#1}}
	\end{tabular*}	
\end{flushleft}	
}

%%% Переопределение отображения счетчиков enumerate на отображение цифрами
\renewcommand{\theenumi}{\arabic{enumi}}
\renewcommand{\labelenumi}{\theenumi)}

