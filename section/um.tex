\section{Настройка УМ}

\subsection{Настройка рабочей точки}

\begin{enumerate}
	\item Резистором R55 насторить напряжение в~рабочей точке равной половине питания. Измерение проводят в контрольной точке <<Uраб.т>>.
	\item Резистором R48 настроить ток в~рабочей точке порядка 50-80~мА (измеренное напряжение в~контрольной точке умноженное на~два). Измерение проводят в~контрольной точке <<Iраб.т>>.	
\end{enumerate}

\subsection{Настройка диф. системы}

\begin{enumerate}
	\item Установить перемычку X11, отключив при~этом преселектор.
	\item Включить УМ на~сигнал со~средней частотой полосы пропускания.
	\item Выставить выходное напряжение порядка 15~В.
	\item Подождать несколько минут.
	\item Резистором R25 (находится внутри металлического экрана) найти минимальный сигнал на~выходе УМ, т.е. на~контактах A2C2 - A1C1.
	\item Убрать перемычку X11.
\end{enumerate}

\subsection{Настройка измерительных цепей}

\begin{enumerate}
	\item Снять плату МкУМ.
	\item Включить УМ на~сигнал со~средней частотой полосы пропускания.
	\item Выставить выходное напряжение порядка 15~В.
	\item Вывести на осциллограф сигнал измерителя напряжения, до~его выпрямления (точка соединения резисторов R27 и~R28).
	\item Проверить что полярность сигнала в~данной точке положительная. Иначе необходимо переместить запаянный резистор в~группе R23 - R24. Например, если был запаян R23, перенести его на~R24. И~после этого повторить этот шаг заново. \label{enum:tune_measure_u1}
	\item Конденсатором C39 добиться максимального прохождения положительной полуволны. \label{enum:tune_measure_u2}
	\item Проверить форму сигнала измерителя тока, до~его выпрямления (точка соединения резисторов R26 и~R79). Если значительная часть сигнала попадает в~область отрицательных значений, добиться компромисса между измерением напряжения и~тока, повторя действия начиная с~шага \ref{enum:tune_measure_u2}. При необходимости, переместить резистор R21 - R22 аналогично пункту \ref{enum:tune_measure_u1}.
	\item Выключить УМ. Установить плату МкУМ.
\end{enumerate}	

\subsection{Подстройка УМ в аппарате}

При установке УМ в аппарат, обычно наблюдается уход резонансной частоты. Поэтому необходимо настроить резонанс на среднюю частоту по входу, т.е. сигнал поданный с генератора на вход УМ должен быть минимальный.